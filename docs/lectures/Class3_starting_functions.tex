% Options for packages loaded elsewhere
\PassOptionsToPackage{unicode}{hyperref}
\PassOptionsToPackage{hyphens}{url}
%
\documentclass[
  ignorenonframetext,
]{beamer}
\title{Starting Functions: Class 3}
\author{}
\date{\vspace{-2.5em}}

\usepackage{pgfpages}
\setbeamertemplate{caption}[numbered]
\setbeamertemplate{caption label separator}{: }
\setbeamercolor{caption name}{fg=normal text.fg}
\beamertemplatenavigationsymbolsempty
% Prevent slide breaks in the middle of a paragraph
\widowpenalties 1 10000
\raggedbottom
\setbeamertemplate{part page}{
  \centering
  \begin{beamercolorbox}[sep=16pt,center]{part title}
    \usebeamerfont{part title}\insertpart\par
  \end{beamercolorbox}
}
\setbeamertemplate{section page}{
  \centering
  \begin{beamercolorbox}[sep=12pt,center]{part title}
    \usebeamerfont{section title}\insertsection\par
  \end{beamercolorbox}
}
\setbeamertemplate{subsection page}{
  \centering
  \begin{beamercolorbox}[sep=8pt,center]{part title}
    \usebeamerfont{subsection title}\insertsubsection\par
  \end{beamercolorbox}
}
\AtBeginPart{
  \frame{\partpage}
}
\AtBeginSection{
  \ifbibliography
  \else
    \frame{\sectionpage}
  \fi
}
\AtBeginSubsection{
  \frame{\subsectionpage}
}
\usepackage{amsmath,amssymb}
\usepackage{lmodern}
\usepackage{iftex}
\ifPDFTeX
  \usepackage[T1]{fontenc}
  \usepackage[utf8]{inputenc}
  \usepackage{textcomp} % provide euro and other symbols
\else % if luatex or xetex
  \usepackage{unicode-math}
  \defaultfontfeatures{Scale=MatchLowercase}
  \defaultfontfeatures[\rmfamily]{Ligatures=TeX,Scale=1}
\fi
% Use upquote if available, for straight quotes in verbatim environments
\IfFileExists{upquote.sty}{\usepackage{upquote}}{}
\IfFileExists{microtype.sty}{% use microtype if available
  \usepackage[]{microtype}
  \UseMicrotypeSet[protrusion]{basicmath} % disable protrusion for tt fonts
}{}
\makeatletter
\@ifundefined{KOMAClassName}{% if non-KOMA class
  \IfFileExists{parskip.sty}{%
    \usepackage{parskip}
  }{% else
    \setlength{\parindent}{0pt}
    \setlength{\parskip}{6pt plus 2pt minus 1pt}}
}{% if KOMA class
  \KOMAoptions{parskip=half}}
\makeatother
\usepackage{xcolor}
\IfFileExists{xurl.sty}{\usepackage{xurl}}{} % add URL line breaks if available
\IfFileExists{bookmark.sty}{\usepackage{bookmark}}{\usepackage{hyperref}}
\hypersetup{
  pdftitle={Starting Functions: Class 3},
  hidelinks,
  pdfcreator={LaTeX via pandoc}}
\urlstyle{same} % disable monospaced font for URLs
\newif\ifbibliography
\usepackage{color}
\usepackage{fancyvrb}
\newcommand{\VerbBar}{|}
\newcommand{\VERB}{\Verb[commandchars=\\\{\}]}
\DefineVerbatimEnvironment{Highlighting}{Verbatim}{commandchars=\\\{\}}
% Add ',fontsize=\small' for more characters per line
\usepackage{framed}
\definecolor{shadecolor}{RGB}{248,248,248}
\newenvironment{Shaded}{\begin{snugshade}}{\end{snugshade}}
\newcommand{\AlertTok}[1]{\textcolor[rgb]{0.94,0.16,0.16}{#1}}
\newcommand{\AnnotationTok}[1]{\textcolor[rgb]{0.56,0.35,0.01}{\textbf{\textit{#1}}}}
\newcommand{\AttributeTok}[1]{\textcolor[rgb]{0.77,0.63,0.00}{#1}}
\newcommand{\BaseNTok}[1]{\textcolor[rgb]{0.00,0.00,0.81}{#1}}
\newcommand{\BuiltInTok}[1]{#1}
\newcommand{\CharTok}[1]{\textcolor[rgb]{0.31,0.60,0.02}{#1}}
\newcommand{\CommentTok}[1]{\textcolor[rgb]{0.56,0.35,0.01}{\textit{#1}}}
\newcommand{\CommentVarTok}[1]{\textcolor[rgb]{0.56,0.35,0.01}{\textbf{\textit{#1}}}}
\newcommand{\ConstantTok}[1]{\textcolor[rgb]{0.00,0.00,0.00}{#1}}
\newcommand{\ControlFlowTok}[1]{\textcolor[rgb]{0.13,0.29,0.53}{\textbf{#1}}}
\newcommand{\DataTypeTok}[1]{\textcolor[rgb]{0.13,0.29,0.53}{#1}}
\newcommand{\DecValTok}[1]{\textcolor[rgb]{0.00,0.00,0.81}{#1}}
\newcommand{\DocumentationTok}[1]{\textcolor[rgb]{0.56,0.35,0.01}{\textbf{\textit{#1}}}}
\newcommand{\ErrorTok}[1]{\textcolor[rgb]{0.64,0.00,0.00}{\textbf{#1}}}
\newcommand{\ExtensionTok}[1]{#1}
\newcommand{\FloatTok}[1]{\textcolor[rgb]{0.00,0.00,0.81}{#1}}
\newcommand{\FunctionTok}[1]{\textcolor[rgb]{0.00,0.00,0.00}{#1}}
\newcommand{\ImportTok}[1]{#1}
\newcommand{\InformationTok}[1]{\textcolor[rgb]{0.56,0.35,0.01}{\textbf{\textit{#1}}}}
\newcommand{\KeywordTok}[1]{\textcolor[rgb]{0.13,0.29,0.53}{\textbf{#1}}}
\newcommand{\NormalTok}[1]{#1}
\newcommand{\OperatorTok}[1]{\textcolor[rgb]{0.81,0.36,0.00}{\textbf{#1}}}
\newcommand{\OtherTok}[1]{\textcolor[rgb]{0.56,0.35,0.01}{#1}}
\newcommand{\PreprocessorTok}[1]{\textcolor[rgb]{0.56,0.35,0.01}{\textit{#1}}}
\newcommand{\RegionMarkerTok}[1]{#1}
\newcommand{\SpecialCharTok}[1]{\textcolor[rgb]{0.00,0.00,0.00}{#1}}
\newcommand{\SpecialStringTok}[1]{\textcolor[rgb]{0.31,0.60,0.02}{#1}}
\newcommand{\StringTok}[1]{\textcolor[rgb]{0.31,0.60,0.02}{#1}}
\newcommand{\VariableTok}[1]{\textcolor[rgb]{0.00,0.00,0.00}{#1}}
\newcommand{\VerbatimStringTok}[1]{\textcolor[rgb]{0.31,0.60,0.02}{#1}}
\newcommand{\WarningTok}[1]{\textcolor[rgb]{0.56,0.35,0.01}{\textbf{\textit{#1}}}}
\setlength{\emergencystretch}{3em} % prevent overfull lines
\providecommand{\tightlist}{%
  \setlength{\itemsep}{0pt}\setlength{\parskip}{0pt}}
\setcounter{secnumdepth}{-\maxdimen} % remove section numbering
\ifLuaTeX
  \usepackage{selnolig}  % disable illegal ligatures
\fi

\begin{document}
\frame{\titlepage}

\begin{frame}{Some conventions (helpful later in the course)}
\protect\hypertarget{some-conventions-helpful-later-in-the-course}{}
\begin{itemize}
\item
  Always write your function in a text editor and then copy into R
\item
  By convention we name files with functions in them by the name of the
  function.R I called my function power\_gen\_orig so I'll save it to
\end{itemize}

e.g.~\textbf{power\_gen\_orig.R}

\begin{itemize}
\item
  you can have R read a text file by source(``power\_gen\_orig.R'') -
  make sure you are in the right working directory
\item
  keep organized by keeping all functions in a subdirectory called
  \emph{R}
\item
  Eventually we will want our function to be part of a package (a
  library of many functions) - to create a package you must use this
  convention (name.R) place all function in a directory called ``R''
\end{itemize}
\end{frame}

\begin{frame}[fragile]{Function Example}
\protect\hypertarget{function-example}{}
\begin{Shaded}
\begin{Highlighting}[]
\CommentTok{\# first set your working directory to just above the R directory}
\FunctionTok{source}\NormalTok{(}\StringTok{"../R/power\_gen\_orig.R"}\NormalTok{)}

\CommentTok{\# to see the code}
\NormalTok{power\_gen\_orig}
\end{Highlighting}
\end{Shaded}

\begin{verbatim}
## function (height, flow, rho = 1000, g = 9.8, Keff = 0.8) 
## {
##     result = rho * height * flow * g * Keff
##     return(result)
## }
\end{verbatim}
\end{frame}

\begin{frame}[fragile]{Some insights into inputs}
\protect\hypertarget{some-insights-into-inputs}{}
A cool thing is that you can use multiple values as inputs to your R
function Lets explore what it can do

\begin{Shaded}
\begin{Highlighting}[]
\CommentTok{\# note that this works (calculates power for each height)}
\FunctionTok{power\_gen\_orig}\NormalTok{(}\AttributeTok{height=}\FunctionTok{c}\NormalTok{(}\DecValTok{1}\NormalTok{,}\DecValTok{4}\NormalTok{,}\DecValTok{5}\NormalTok{), }\AttributeTok{flow=}\DecValTok{2}\NormalTok{)}
\end{Highlighting}
\end{Shaded}

\begin{verbatim}
## [1] 15680 62720 78400
\end{verbatim}

\begin{Shaded}
\begin{Highlighting}[]
\CommentTok{\# and this works (calculates power for each flow rate)}
\FunctionTok{power\_gen\_orig}\NormalTok{(}\AttributeTok{height=}\DecValTok{4}\NormalTok{, }\AttributeTok{flow=}\FunctionTok{c}\NormalTok{(}\DecValTok{2}\NormalTok{,}\DecValTok{4}\NormalTok{,}\DecValTok{5}\NormalTok{))}
\end{Highlighting}
\end{Shaded}

\begin{verbatim}
## [1]  62720 125440 156800
\end{verbatim}

\begin{Shaded}
\begin{Highlighting}[]
\CommentTok{\# and this calculates power for each combination of flow and height}
\FunctionTok{power\_gen\_orig}\NormalTok{(}\AttributeTok{height=}\FunctionTok{c}\NormalTok{(}\DecValTok{4}\NormalTok{,}\DecValTok{5}\NormalTok{,}\DecValTok{6}\NormalTok{), }\AttributeTok{flow=}\FunctionTok{c}\NormalTok{(}\DecValTok{2}\NormalTok{,}\DecValTok{4}\NormalTok{,}\DecValTok{6}\NormalTok{))}
\end{Highlighting}
\end{Shaded}

\begin{verbatim}
## [1]  62720 156800 282240
\end{verbatim}

\begin{Shaded}
\begin{Highlighting}[]
\CommentTok{\# but this doesn\textquotesingle{}t work {-} why?}
\FunctionTok{power\_gen\_orig}\NormalTok{(}\AttributeTok{height=}\FunctionTok{c}\NormalTok{(}\DecValTok{4}\NormalTok{,}\DecValTok{5}\NormalTok{,}\DecValTok{6}\NormalTok{, }\DecValTok{8}\NormalTok{, }\DecValTok{20}\NormalTok{), }\AttributeTok{flow=}\FunctionTok{c}\NormalTok{(}\DecValTok{2}\NormalTok{,}\DecValTok{4}\NormalTok{,}\DecValTok{6}\NormalTok{))}
\end{Highlighting}
\end{Shaded}

\begin{verbatim}
## Warning in rho * height * flow: longer object length is not a multiple of
## shorter object length
\end{verbatim}

\begin{verbatim}
## [1]  62720 156800 282240 125440 627200
\end{verbatim}
\end{frame}

\begin{frame}{}
\protect\hypertarget{section}{}
Note that by \emph{sourcing} a function - it will essentially overwrite
anything else in your workspace with the same name

\emph{Warning} - if you name your function, the same as an internal R
function, your new function will take precidence, and \textbf{hide} the
internal R function

Ideally choose function names that are likely to be unique

If you overwrite a \emph{base} package R function - good to delete it
(see below)

You can also use \emph{package\_name}:: to get to a function that is
hidden in the current workspace
\end{frame}

\begin{frame}[fragile]{}
\protect\hypertarget{section-1}{}
\begin{Shaded}
\begin{Highlighting}[]
\CommentTok{\# hiding the sort function}

\CommentTok{\# normal use}
\NormalTok{mydata }\OtherTok{=} \FunctionTok{c}\NormalTok{(}\DecValTok{1}\NormalTok{,}\DecValTok{5}\NormalTok{,}\DecValTok{8}\NormalTok{,}\DecValTok{22}\NormalTok{,}\DecValTok{2}\NormalTok{)}
\FunctionTok{sort}\NormalTok{(mydata)}
\end{Highlighting}
\end{Shaded}

\begin{verbatim}
## [1]  1  2  5  8 22
\end{verbatim}

\begin{Shaded}
\begin{Highlighting}[]
\CommentTok{\# define a new function that adds two numbers {-} silly {-} but lets call it sort}
\NormalTok{sort }\OtherTok{=} \ControlFlowTok{function}\NormalTok{(a,b) \{}
\NormalTok{  add }\OtherTok{=}\NormalTok{ a}\SpecialCharTok{+}\NormalTok{b}
  \FunctionTok{return}\NormalTok{(add)}
\NormalTok{\}}

\CommentTok{\# now run}
\FunctionTok{sort}\NormalTok{(mydata)}
\end{Highlighting}
\end{Shaded}

\begin{verbatim}
## Error in sort(mydata): argument "b" is missing, with no default
\end{verbatim}

\begin{Shaded}
\begin{Highlighting}[]
\CommentTok{\# how do we get the base package sort back}
\NormalTok{base}\SpecialCharTok{::}\FunctionTok{sort}\NormalTok{(mydata)}
\end{Highlighting}
\end{Shaded}

\begin{verbatim}
## [1]  1  2  5  8 22
\end{verbatim}

\begin{Shaded}
\begin{Highlighting}[]
\CommentTok{\# or get rid of our silly function in the workspace}
\FunctionTok{rm}\NormalTok{(sort)}
\FunctionTok{sort}\NormalTok{(mydata)}
\end{Highlighting}
\end{Shaded}

\begin{verbatim}
## [1]  1  2  5  8 22
\end{verbatim}
\end{frame}

\begin{frame}{Documentation Best Practices}
\protect\hypertarget{documentation-best-practices}{}
In additon to \emph{in-line} documents - at the top of the * *.R * where
you define you function, include information about the functions
contract

\begin{itemize}
\tightlist
\item
  descriptions
\item
  inputs and parameters
\item
  outputs
\end{itemize}

With units and default values included

See example \emph{R/power\_gen\_orig.R}
\end{frame}

\begin{frame}{{Documentation}}
\protect\hypertarget{documentation}{}
\begin{itemize}
\item
  There is a standard format for documentation that can be read by
  automatic programs (roxygen2) - an R package that generate
  ``standard'' R documentation - manual or help pages
\item
  These automated approaches for building documentation (like roxygen2)
  and meta data (descriptions of data sets) are increasingly common - so
  you should get into the practice of being structured in your approach
  to documentation
\item
  We will use the conventions that work with roxygen2 - and then use
  this program to generate formal R documentation. Roxygen is similar to
  Doxygen which is used for C code\ldots so its a widely used format
\end{itemize}

Documentation is placed at the top of the \emph{something.R} file all
lines start with \emph{\#'}
\end{frame}

\begin{frame}{{Documentation Components}}
\protect\hypertarget{documentation-components}{}
Later in the course we will make our own package and create help pages,
you can use the documentation at the top of your function definitions
for this - so lets get into the \emph{habit} now

Two parts

\begin{itemize}
\tightlist
\item
  \textbf{Description} - summary of what your model/function does
\item
  \textbf{Tagged (using special ``key'' words)}
\end{itemize}

Here are some examples there are many others

\begin{itemize}
\tightlist
\item
  \emph{@param} inputs/parameter description
\item
  \emph{@return} what your function returns (outputs)
\item
  \emph{@example} how to use it
\item
  \emph{@references} citations or urls
\item
  \emph{@author} YOU
\end{itemize}

(you don't need all of these and there are more tags, but start with at
least param and return, example is a good idea

See example
\end{frame}

\begin{frame}{{ Lets practice:: Automobile Power Generation }}
\protect\hypertarget{lets-practice-automobile-power-generation}{}
Some background

One of the equations used to compute automobile fuel efficiency is as
follows this is the power required to keep a car moving at a given speed
is as follows:

\(Pb = crolling * m *g *V + \frac{1}{2} * A*p_{air}*c_{drag}*V^3\)

where crolling and cdrag are rolling and aerodynamic resistive
coefficients, typical values are 0.015 and 0.3, respectively.

\begin{itemize}
\item
  V: is vehicle speed (assuming no headwind) in m/s (or mps) m: is
  vehicle mass in kg
\item
  A is surface area of car \((m^2)\)
\item
  \emph{g}: is acceleration due to gravity \((9.8 m/s^2 )\)
\item
  p\_air = density of air \((1.2kg/m^3 )\)
\item
  Pb is power in Watts
\end{itemize}

What would be the function's contract - input/parameters? what will it
do?

Write a function to compute power, given a truck of m=31752 kg
(parameters for a heavy truck) for a range of different highway speeds
(30 m/s, 80 km/hr, 120 km/hr, 40 km/hr, 5km/hr ) The surface area is 16
\(m^2\)

Plot power as a function of speed Add a second line for lighter vehicle
Remember to keep the function in its own file, store it in a
subdirectory called \emph{R} and then create a Rmarkdown to use the
function and store the Rmarkdown in a different directory

\textbf{Good Programming Practices}

Name variables so its easy to understand

Describe the contract at the top of the function (following the format
for the \emph{power\_gen\_orig.R} example

Also include the comments in the Rmarkdown that you are using to run the
function
\end{frame}

\begin{frame}{{ PART 2 - Review Two Rmarkdowns }}
\protect\hypertarget{part-2---review-two-rmarkdowns}{}
\begin{itemize}
\item
  Sampling\_example.Rmd
\item ~
  \hypertarget{error_checking.rmd}{%
  \subsection{Error\_checking.Rmd}\label{error_checking.rmd}}
\end{itemize}
\end{frame}

\end{document}
